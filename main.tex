\documentclass{article}
\usepackage[utf8]{inputenc}
\usepackage[spanish]{babel}
\usepackage{listings}
\usepackage{graphicx}
\graphicspath{ {images/} }
\usepackage{cite}

\begin{document}

\begin{titlepage}
    \begin{center}
        \vspace*{1cm}
            
        \Huge
        \textbf{Experimento}
            
        \vspace{0.5cm}
        \LARGE
        Mover un objeto con una mano
            
        \vspace{1.5cm}
            
        \textbf{Mario Andrés Leal Galvis}
            
        \vfill
            
        \vspace{0.8cm}
            
        \Large
        Despartamento de Ingeniería Electrónica y Telecomunicaciones\\
        Universidad de Antioquia\\
        Medellín\\
        Marzo de 2021
            
    \end{center}
\end{titlepage}

\tableofcontents

\section{Instrucciones}
Este experimento se realizara con una mano, los objetos a utilizar son una hoja de papel y 2 tarjetas plastificadas.Se debe seguir las instrucciones paso a paso.

\section{Pasos} \label{contenido}

1)coloca la hoja acostada de forma horizontal

2)toma con tu mano un borde de la hoja y llévelo al otro extremo de la hoja

3)sin soltar el dobles alinee perfectamente las esquinas de tal forma que quede un extremo de la hoja sobre el otro extremo

4)con tu palma de la mano trata de hacerle fuerza a la hoja sobre la mesa

5)suelte la hoja y gírela 90 grados y repita los pasos 2,3,4,5

6)donde se unen las marcas que dejaron los procesos anteriores será el centro de la hoja

7)alce con la mano la tarjeta, lleva la tarjeta sobre la hoja y coloque la tarjeta sobre el centro de la hoja 

8)repita el proceso anterior pero con la otra tarjeta y suelte esta tarjeta sobre la anterior tarjeta

9)tome con una mano las tarjetas

10)empuñe un poco la mano con las tarjetas tomadas, ten en cuenta que la acción empuñar es llevar tus dedos hacia el centro de su mano

11)sostén las tarjetas sobre la hoja

12)has que las 2 tarjetas toquen el centro de la hoja pero por la parte mas corta de las tarjetas

13)ahora con uno de sus dedos y sin soltar las tarjetas vaya separando las tarjetas lentamente de tal forma que forme una "V"  al revés

14)pruebe si las tarjetas se quedan paradas juntas, para esto suelta las tarjetas después de haber abierto un poco las tarjetas 

14)en caso de que falle, repita los pasos 7,8,9,10,11,12,13,14



\section{Conclusión} \label{conclulsion}
Fue un experimento muy divertido debido a que nos dimos cuenta que dejamos mucho a la imaginacion, en caso de la programacion se debe ser muy especifico en todos los procesos ya que el computador no puede razonar por el mismo

\end{document}
